\documentclass[]{article}
\usepackage{lmodern}
\usepackage{amssymb,amsmath}
\usepackage{ifxetex,ifluatex}
\usepackage{fixltx2e} % provides \textsubscript
\usepackage{fontspec}
\usepackage{unicode-math}

\usepackage[a4paper,
            bindingoffset=0.2in,
            left=0.5in,
            right=0.5in,
            top=0.5in,
            bottom=0.5in,
            footskip=.25in]{geometry}

\usepackage{blindtext}

% use upquote if available, for straight quotes in verbatim environments
\IfFileExists{upquote.sty}{\usepackage{upquote}}{}
% use microtype if available
\IfFileExists{microtype.sty}{%
\usepackage{microtype}
\UseMicrotypeSet[protrusion]{basicmath} % disable protrusion for tt fonts
}{}
\usepackage[unicode=true]{hyperref}
\hypersetup{
            pdfborder={0 0 0},
            breaklinks=true}
\urlstyle{same}  % don't use monospace font for urls
\usepackage{longtable,booktabs}
\IfFileExists{parskip.sty}{%
\usepackage{parskip}
}{% else
\setlength{\parindent}{0pt}
\setlength{\parskip}{6pt plus 2pt minus 1pt}
}
\setlength{\emergencystretch}{3em}  % prevent overfull lines
\providecommand{\tightlist}{%
  \setlength{\itemsep}{0pt}\setlength{\parskip}{0pt}}
\setcounter{secnumdepth}{0}
% Redefines (sub)paragraphs to behave more like sections
\ifx\paragraph\undefined\else
\let\oldparagraph\paragraph
\renewcommand{\paragraph}[1]{\oldparagraph{#1}\mbox{}}
\fi
\ifx\subparagraph\undefined\else
\let\oldsubparagraph\subparagraph
\renewcommand{\subparagraph}[1]{\oldsubparagraph{#1}\mbox{}}
\fi

\date{}

\begin{document}

\begin{minipage}{0.45\textwidth}
\begin{center}
ĐOÀN KHOA CÔNG NGHỆ THÔNG TIN

\textbf{BCH CHI ĐOÀN 21CLC02}

\textbf{***}
\end{center}

\end{minipage}%
\hfill
\begin{minipage}{0.45\textwidth}
\begin{tabular}{p{\textwidth}}
\begin{center}
{\fontsize{14}{20pt}\textbf{\underline{ĐOÀN TNCS HỒ CHÍ MINH}}}

\vspace{\baselineskip}

\emph{TP. Hồ Chí Minh, ngày \ldots{} tháng \ldots{} năm 2023}\ 
\end{center}
\end{tabular}
\end{minipage}%

\begin{center}
{\fontsize{14}{20pt} \textbf{BIÊN BẢN HỌP CHI ĐOÀN}}

\textit{\textbf{V/v đề nghị xét chuyển Đảng chính thức cho Đảng viên dự bị
Hồ Thanh Nhân}}

\textbf{-\/-\/-\/-\/-\/-\/-\/-\/-\/-\/-}
\end{center}
\textbf{I. THỜI GIAN, ĐỊA ĐIỂM, THÀNH PHẦN THAM DỰ}

\textbf{1. Thời gian:} \ldots{}  h  \ldots{} ngày \ldots{} tháng \ldots{}
năm 2023.

\textbf{2. Địa điểm:} Phòng học I23, Trường Đại học Khoa học tự nhiên,
ĐHQG-HCM, Cơ sở 1.

\textbf{3. Thành phầm tham dự:}

\begin{quote}
- Đại diện Chi uỷ Chi bộ Sinh viên 5: Đồng chí Nguyễn Phúc Thuần -- Đảng
viên chính thức của Chi ủy Chi bộ Sinh viên 5.

- Đại diện Đoàn khoa: Đồng chí Lê Hình Nhựt Thanh -- Bí thư Đoàn khoa
Công nghệ thông tin.
\end{quote}

- Ban Chấp hành chi Đoàn 21CLC02: có mặt 4/5 (chiếm 80\%).

\textbf{II. NỘI DUNG}

\begin{quote}
Ban chấp hành chi Đoàn 21CLC02 tiến hành nhận xét cho Đảng viên dự bị
\emph{\textbf{Hồ Thanh Nhân}} trên các nội dung:
\end{quote}

\emph{Ưu điểm: }

\begin{itemize}
\item
  Về tư tưởng, tổ chức, chính trị: Đảng viên dự bị Hồ Thanh Nhân có tư
  tưởng chính trị vững vàng, luôn tuân thủ đường lối, chủ trương của
  Đảng, Nhà nước và các cấp ủy. Đảng viên dự bị thường xuyên học tập,
  nghiên cứu các văn kiện của Đảng, các quy định của pháp luật, các kiến
  thức chuyên môn và nâng cao năng lực lãnh đạo, quản lý. Đảng viên dự
  bị có tinh thần tự giác, kỷ luật, chấp hành nghiêm túc các quy định
  của Đảng, tổ chức và nơi làm việc.
\item
  Về phẩm chất đạo đức, lối sống: Đảng viên dự bị Hồ Thanh Nhân có đạo
  đức tốt, lối sống lành mạnh, không có những hành vi vi phạm pháp luật
  hay đạo đức xã hội. Đảng viên dự bị có tinh thần trách nhiệm cao, làm
  việc nghiêm túc, chuyên nghiệp, có tinh thần hợp tác, giúp đỡ đồng
  nghiệp và quần chúng. Đảng viên dự bị biết tự kiểm điểm, nhận xét, phê
  bình và khắc phục những sai sót trong công việc và sinh hoạt Đảng.
\item
  Về chuyên môn: Đảng viên dự bị Hồ Thanh Nhân có trình độ chuyên môn
  cao, nắm vững kiến thức và kỹ năng trong lĩnh vực của mình. Đảng viên
  dự bị có nhiều sáng kiến, đóng góp tích cực cho sự phát triển của đơn
  vị và tổ chức. Đảng viên dự bị luôn cập nhật, học hỏi những kiến thức
  mới, áp dụng những tiến bộ khoa học kỹ thuật vào công việc.
\item
  Về mối quan hệ quần chúng: Đảng viên dự bị Hồ Thanh Nhân có mối quan
  hệ tốt với quần chúng, được quần chúng tin tưởng, yêu mến. Đảng viên
  dự bị luôn lắng nghe, thấu hiểu và giải quyết kịp thời những khó khăn,
  vướng mắc của quần chúng. Đảng viên dự bị có tinh thần đoàn kết, hợp
  tác, chia sẻ với các đoàn thể khác, góp phần xây dựng một môi trường
  làm việc hòa bình, thân thiện.
\item
  Đối với các hoạt động của đoàn thể và hoạt động đoàn: Đảng viên dự bị
  Hồ Thanh Nhân luôn tích cực tham gia các hoạt động của đoàn thể và
  hoạt động đoàn, đóng góp ý kiến, kiến nghị cho các vấn đề liên quan
  đến đoàn thể và hoạt động đoàn. Đảng viên dự bị có tinh thần đồng tâm,
  đồng lòng, đồng hành với các đoàn viên, thanh niên khác trong các hoạt
  động xã hội, văn hóa, thể thao, tình nguyện.
\end{itemize}

\emph{Hạn chế: }

\begin{itemize}
\item
  Đảng viên dự bị Hồ Thanh Nhân còn có một số hạn chế như: chưa thể hiện
  được sự lãnh đạo, đi đầu trong một số hoạt động của đoàn thể và hoạt
  động đoàn; chưa chủ động liên hệ, trao đổi với các cấp ủy và các đoàn
  thể khác để học hỏi kinh nghiệm và phối hợp công tác; chưa có nhiều
  hoạt động thiết thực để phục vụ quần chúng và xã hội.
\end{itemize}

\textbf{III. KẾT QUẢ BIỂU QUYẾT}

Căn cứ vào những tiêu chí phấn đấu của đảng viên Đảng Cộng Sản Việt Nam,

\begin{itemize}
\item
  Đồng ý xét chuyển Đảng chính thức cho đ/c Hồ Thanh Nhân: 4/5 đ/c (tỉ
  lệ 80\%).
\item
  Không đồng ý xét chuyển Đảng chính thức cho đ/c Hồ Thanh Nhân: 0/5 đ/c
  (tỉ lệ 0\%).
\end{itemize}


\begin{minipage}{0.45\textwidth}
\begin{center}
\textbf{CHỦ TỌA}
\end{center}

\end{minipage}%
\hfill
\begin{minipage}{0.45\textwidth}
\begin{tabular}{p{\textwidth}}
\begin{center}
\textbf{THƯ KÝ}
\end{center}
\end{tabular}
\end{minipage}%

\end{document}
